%% Copernicus Publications Manuscript Preparation Template for LaTeX Submissions%% ---------------------------------%% This template should be used for copernicus.cls
%% The class file and some style files are bundled in the Copernicus Latex Package which can be downloaded from the different journal webpages.%
% For further assistance please contact the Copernicus Publications at: publications@copernicus.org%% http://publications.copernicus.org%% Please use the following documentclass and Journal Abbreviations for Discussion Papers and Final Revised Papers.%% 2-Column Papers and Discussion Papers\documentclass[11pt, onecolumn]{IEEEtran} %
% choose manuscript,article or draft%% Journal Abbreviations (Please use the same for Discussion Papers and Final Revised Papers)% Archives Animal Breeding (aab)
% Atmospheric Chemistry and Physics (acp)% Advances in Geosciences (adgeo)% Advances in Statistical Climatology, Meteorology and Oceanography (ascmo)
% Annales Geophysicae (angeo)% ASTRA Proceedings (ap)% Atmospheric Measurement Techniques (amt)% Advances in Radio Science (ars)% Advances in Science and Research (asr)% Biogeosciences (bg)% Climate of the Past (cp)% Drinking Water Engineering and Science (dwes)% Earth System Dynamics (esd)% Earth Surface Dynamics (esurf)% Earth System Science Data (essd)% Fossil Record (fr)% Geographica Helvetica (gh)% Geoscientific Instrumentation, Methods and Data Systems (gi)
% Geoscientific Model Development (gmd)% Geothermal Energy Science (gtes)% Hydrology and Earth System Sciences (hess)% History of Geo- and Space Sciences (hgss)
% Journal of Sensors and Sensor Systems (jsss)% Mechanical Sciences (ms)% Natural Hazards and Earth System Sciences (nhess)% Nonlinear Processes in Geophysics (npg)
% Ocean Science (os)% Proceedings of the International Association of Hydrological Sciences (piahs)% Primate Biology (pb)% Scientific Drilling (sd)
% SOIL (soil)
% Solid Earth (se)% The Cryosphere (tc)% Web Ecology (we)% Wind Energy Science (wes)%% \usepackage commands included in the copernicus.cls:



\usepackage[german, english]{babel}
\usepackage{tabularx}
\usepackage{cancel}
\usepackage{multirow}
\usepackage{supertabular}
\usepackage{algorithmic}
\usepackage{algorithm}
\usepackage{amsthm}
\usepackage{float}
\usepackage{subfig}
\usepackage{rotating}
\usepackage{amssymb}
\usepackage{color}
\usepackage{graphicx}
\usepackage{amsmath}
\usepackage{wasysym}
\usepackage{multirow}
\usepackage{booktabs}
\usepackage{lineno}
\usepackage{soul}%
\usepackage{subcaption}
\usepackage[draft]{fixme}
\usepackage{longtable}
\usepackage{supertabular}
\usepackage{authblk}
\usepackage[]{natbib}


\newcommand{\delOx}{$\delta{}^{18}\mathrm{O}$ }
\newcommand{\delD}{$\delta\mathrm{D}$ }
\newcommand{\Dxs}{$\mathrm{D_{xs}}$ }
\newcommand{\delN}{$\delta{}^{15}\mathrm{N}$ }
\newcommand{\degC}{${}^{\circ}\mathrm{C}$ }
\newcommand{\delAr}{$\delta{}^{40}\mathrm{Ar}$ }

\linenumbers

\begin{document}

\title{An assessment of diffusion-based temperature proxies}

% \Author[affil]{given_name}{surname}
\author[1]{C. Holme}
\author[1]{V. Gkinis}
\author[1]{B. M. Vinther}\affil[1]{The Niels Bohr Institute, Centre for 
Ice and Climate, Juliane Maries Vej 30, 2100 Copenhagen, Denmark}


\maketitle

\begin{abstract}Polar precipitation archived in ice cores contains paleoclimate information. This study examines different varieties of 
diffusion-based temperature proxies, which are dependent on the amount of firn diffusion influencing the water isotopes $\delta^{18}$O and $\delta$D in polar ice cores. 
Dual isotope measurements of $\delta^{18}$O and $\delta$D enable multiple temperature proxies and different ways of deriving these diffusion-based proxies are here 
outlined and applied on various ice cores from Greenland and Antarctica. The selected cores represent a wide temperature and accumulation rate span together with 
varying sampling size which serve to assess the significance of the results.\\The results indicate that the single isotope diffusion lengths $\sigma_{18}^2$ and $
\sigma_D^2$ are the most precise and accurate of all the proxies, and that the $\sigma_{18}^2/\sigma_D^2$ proxy seems to be the most unreliable. A large part of this 
instability is attributed to the fractionation factor parametrization, which has a significant impact on the $\sigma_{18}^2/\sigma_D^2$ proxy. The differential diffusion length 
proxy $\Delta\sigma^2$ have larger uncertainties than the $\sigma_{18}^2$ and $\sigma_D^2$ proxies, but the independence from sampling and ice diffusion makes this 
proxy potentially suitable in determining the amplitudes of climate transitions. Diffusion-based paleothermometry are found to be a reliable temperature proxy for polar ice 
cores.\end{abstract}

%%%%%%%%%%%%%%%%%%%%%%%%%%%%%%%%%%%%%%%%%%%%%%%%%%%%%%%%%%%%%%%%%%%%%%%%%%
%%%%%%%%%%%%%%%%%%%%%%%%%%

\section{Introduction} 
%% \introduction[modified heading if necessary]
\label{Intro}Polar precipitation stored for 
thousands of years at the  ice caps of Greenland and Antarcticacontains invaluable information on past climatic conditions. The isotopic composition of polar ice  
commonly expressedthrough the $\delta$ notationhas been shown to work as a direct proxy of the relative depletion of a water vapour mass in its journey from the 
evaporation site to the place where condensation takes place \citep{Epstein1951, IAEA}. Additionally, for modern times, it shows a good correlation with the temperature 
of the cloud at thetime of precipitation \citep{Dansgaard, Dansgaard2} and as a result it has been proposed and used as a proxy of past temperatures \citep{Jouzel1984, Jouzel1997, Johnsen2001}. 
The use of the isotopic paleothermometer presents however some notable limitations.The linear relationship between \delOx and temperature 
commonly referred to as the ``spatial slope''may hold for present conditions. However previous studies based on more physical principles as the borehole temperature 
reconstruction \citep{Cuffey1994, JOHNSEN1995a} as well as the thermal fractionation of the {\delN} signal in polar firn \citep{Schwander1988, Severinghaus1998, 
Severinghaus1999} haveindependently underlined the inaccuracy of the spatial isotope slope when it is extrapolated to past climatic conditions. Even though qualitatively 
the {\delOx} signal comprises past temperature information, it failsto provide a quantitative picture on the magnitudes of past climatic changes.In a seminal paper in 2000, 
\cite{Johnsen2000} set the foundations for the quantitative description of the diffusive processes the water isotopic signal undergoes in the porous firn layer from the 
timeof deposition until pore close--off. Even though the main purpose of that work was to investigatehow to reconstruct the part of the signal that was attenuated during 
the diffusive processes, the authorsmake a clear reference to the possibility of using the assessment of the diffusive rates as a proxy forpast firn temperatures. The 
temperature reconstruction method based on isotope firn diffusion requires data of high resolution. More specifically, if one would like to look into the the differential 
diffusion signal, datasets of both {\delOx} and {\delD} are required. Such datasets have until recently not been easy toobtain especially due to the challenging nature of 
the {\delD} analysis \citep{BIGELEISEN1952, Vaughn1998}.Nowadays, with the advent of commercial high--accuracy, high--precision Infra-Red spectrometers 
\citep{Crosson2008, Brand2009}, simultaneous measurements of {\delOx} and {\delD} have become easier to obtain. Coupling of these instruments to Continuous Flow 
Analysis systems \citep{Gkinis2011, Emanuelsson2015} can also result in measurements of ultra--high resolution, a necessary condition for accurate temperature 
reconstructions based on water isotope diffusion.A number of existing works have presented past firn temperature reconstructionsbased on water isotope diffusion. 
\cite{Simonsen2011} and \cite{Gkinis2014} used high resolution isotopic datasets from the NorthGRIP ice core \citep{NGRIPmembers2004}. The first study makes useof 
the differential diffusion signal, utilising spctral estimates of high--resolution dual  {\delOx} and {\delD} datasets coveringthe GS--1 and GI--1 periods in the NorthGRIP ice 
core \citep{Rasmussen2014}. The second study presents a combined temperature and accumulation history of the past 16,000 years based on the power spectral density 
signals of high resolution {\delOx} measurements of the NorthGRIP ice core. More recently, \cite{vanderWel2015a}introduced a slightly different approach for 
reconstructing the differential diffusion signal and testing iton dual {\delOx}, {\delD} high resolution data from the EDML ice core \citep{Oerter2004}. By artificially forward--
diffusing the {\delD} signal the authors estimate differential diffusion ratesby maximising the correlation between the {\delOx} and {\delD} signal.  In this work we attempt to 
look into all different flavours of temperature diffusionreconstruction techniques and assess their performance. We use synthetic, as well as real ice core data sets that 
represent Holocene conditions from a variety of drilling sites on Greenland and Antarctica.Our objective is to use data sections that originate from parts of the core as 
close to present day as possible. By doing this we aim to minimise possible uncertainties and biases in the ice flow thinningadjustment that is required for temperature 
interpretation of the diffusion rate estimates.Such a bias has been shown to exist for the NorthGRIP ice core \citep{Gkinis2014}, most likely due to the Dansgaard--
Johnsen ice flow model overestimating  the past accumulation rates for the site. For some cases however this was not possible and approximately half of the datasets 
used herehave an age at approximately around Holocene Climate Optimum (\textbf{HCO} hereafter). Another interesting aspect of this study is that it uses water isotopic 
data sets of {\delOx} and {\delD}measured using different analytical techniques, namely Isotope Ratio Mass Spectroscopy (\textbf{IRMS} hereafter) as  well as Cavity 
Ring Down Spectroscopy (\textbf{CRDS} hereafter).Two of the data sets presented here were obtained using Continuous Flow Analysis (\textbf{CFA} hereafter)systems 
tailored for water isotopic analysis \citep{Gkinis2011}. All data sections are characterised bya very high sampling resolution typically of 5 cm or better. 
%Ice cores contain information about the paleoclimate as it archives past precipitation. This is a result of the precipitated snow containing a temperature signal due to the difference between the heavy (\chem{^1 H D^{16}O} and \chem{^1H_2^{18}O}) and light (\chem{^1H_2^{16}O}) water stable isotopologues being traceable from the evaporation site in the subtropics to the precipitation site \citep{Dansgaard,Dansgaard2}. The difference between the abundant and rare isotopes' deviation from a reference value are expressed as:%%\begin{align}%&\delta^{18}\mathrm{O} = \frac{\frac{^{18}\mathrm{O}}{^{16}\mathrm{O}} - \left[\frac{^{18}\mathrm{O}}{^{16}\mathrm{O}}\right]_{ref}}{ \left[\frac{^{18}\mathrm{O}}{^{16}\mathrm{O}}\right]_{ref}}  \cdot 1000 \permil \\ %&\delta \mathrm{D} = \frac{\frac{\mathrm{D}}{^{1}\mathrm{H}} - \left[\frac{\mathrm{D}}{^{1}\mathrm{H}}\right]_{ref}}{ \left[\frac{\mathrm{D}}{^{1}\mathrm{H}}\right]_{ref}}  \cdot 1000 \permil%\end{align}%%The $\delta^{18}\mathrm{O}$ and $\delta$D signals in ice cores are strongly correlated with the temperature at the drill site in the water droplet condensation height \citep{Dansgaard, Dansgaard2}. A linear relationship between the annual surface temperature at the drill site and the annually measured $\delta^{18}$O of the isotopes based on Greenland ice cores was given in \citet{Johnsen1989}. However, this relationship only contained a spatial slope, and it did not apply for glacial records \citep{Johnsen2001}. \\ %This study examines temperature proxies which are based on the amount of attenuation experienced by the isotopes after the deposition, hence, the methods are independent of the spatial and temporal slopes. The theory was proposed in \citet{Johnsen2000} and successfully used for the NGRIP core in \citet{Simonsen2011,Gkinis2014}. The objective of this study is to examine the performance of the diffusion-based temperature proxies. This is done by examining different ways of estimating the diffusion length of the water stable isotopes applied on different ice cores. Moreover, dual isotope measurements of the different ice cores facilitates the investigation of other temperature proxies based on the difference in the diffusive properties of the water isotopologues.\\%The underlying principle is that higher temperatures induce higher rates of diffusive smoothing of the isotopic signal. Thus, it is possible to find a temperature that minimizes the difference between model based and data based diffusion lengths.\\
%The selected ice cores have different drill site characteristics such as surface temperature, surface pressure, thinning, sampling and accumulation rate in order to have a wide appliance. This will help assess the significance of the results.%%%%%%%%%%%%%%%%%%%%%%%%%%%%%%%%%%%%%%%%%%%%%%%%%%
%%%%%%%%%%%%%%%%%%%%%%%%%%%%%%%%%%%%%%%%%%%%%%%%%%

\section{Diffusion of water isotope signals in firn}
The main focus of 
this section is to outline the various temperature reconstruction techniques that can possibly be employed for paleotemperature reconstructions.The fundamentals of 
isotope diffusion theory are also presented. In order to avoid significant overlap  with previously published works that have dealt with the matter e.g. \citep{Johnsen1977, 
Johnsen2000, Simonsen2011, Gkinis2014, vanderWel2015a} we occasionally  point the reader to any of the latter or/and refer to specific sections in the Supplementary 
Online Material (\textbf{SOM} hereafter).We exemplify and illustrate the use of various techniques using synthetic data prepared such that they resemble  two 
representative regimes of ice coring sites on the Greenland summit and the East Antarctic Plateau. The porous medium of the top 60-80 m of firn allows for a molecular 
diffusion process that attenuatesthe water isotope signal from the time of deposition until  pore close--off. The process takes place in thevapour phase and it can be 
described by Fick's second law as:\begin{equation}\frac{\partial \delta}{\partial t} = D \left( t \right) \frac{\partial^2 \delta}{\partial z^2} - \dot{\varepsilon}_z \left( t \right) z ~
\frac{\partial \delta}{\partial z} \enspace \label{eq.diffusion}\end{equation}where $z$ is the depth from surface, $\delta$ refers to the water isotope ratio signal,$D \left( t 
\right)$ is the diffusivity coefficient and $\dot{\varepsilon}_z\left(t\right)$the vertical strain rate.  The attenuation of the isotopic signal results in loss of information.However 
the dependance of $\dot{\varepsilon}_z\left(t\right)$  and $D \left( t \right)$ on temperature and accumulation presents the possibility of using the process as a toolto infer 
these two paleoclimatic parameters. A solution to eq. \ref{eq.diffusion} can be given bythe convolution of the initial isotopic profile $\delta'$ with a Gaussian filter $
\mathcal{G}$ as:\begin{equation}\delta \left( z\right) = \mathcal{S} \left( z \right) \left[ \delta ' \left( z \right) \ast \mathcal{G} \left( z \right) \right]\label{eq.convolution}
\end{equation}where the Gaussian filter is described as:\begin{equation}\mathcal{G}\left(z \right) = \frac{1}{\sigma \sqrt{2\pi}} \, e^{\frac{-z^2}{2 \sigma^2}} \enspace ,
\label{eq.gaussian}\end{equation}and $\mathcal{S}$ is the total thinning of the layer at depth $z$ described by\begin{equation}\mathcal{S} \left( z \right) = e^{\int_0^{z} 
\dot{\varepsilon}_z \left( z' \right) \mathrm{d}z'} \enspace .\label{eq.thinning}\end{equation}where  $\dot{\varepsilon}_z \left( z' \right)$ is the vertical strain rate.In eq. 
\ref{eq.gaussian}, the standard deviation term $\sigma^2$ represents the average displacement of a water molecule along the z--axis and is commonly referred to as the 
diffusion length.The $\sigma^2$ quantity is a direct measure of diffusion and its accurate estimate is criticalto any attempt of reconstructing temperatures that are based 
on the isotope diffusion thermometer.  Since it is directly related to the diffusivity coefficient and the strain rate -the latter for the case of the firn being approximately 
proportional to the densification rate-, it can be regarded as a sensor of firn temperature as well as accumulation rate.  The differential equation describing the evolution of 
$\sigma^2$ with time  can be given by \citep{Johnsen1977}: \begin{equation}\frac{\mathrm{d}\sigma^2}{\mathrm{d}t} - 2\,\dot{\varepsilon}_z\!\left( t \right) \sigma^2 = 2D\!
\left( t \right) \enspace .\label{eq.diflength}\end{equation}In the case of firn the following approximation can be made for the strain rate:\begin{equation}\dot{\varepsilon}_z 
\left(	t \right) \approx -\frac{\mathrm{d\rho}}{\,\,\,\mathrm{d t}}\,\frac{\,1\,}{\,\rho\,}.\label{eq.strainrate}\end{equation}Then we can solve eq. \ref{eq.diflength}  obtaining a 
solution for $\sigma^2$:\begin{equation}\sigma^2 \left( \rho \right) = \frac{\,1\,}{\rho^2}\int_{\rho_o}^{\rho}2\rho^2 {\left( \frac{\mathrm{d}\rho}{\mathrm{d}t}\right)}^{-1}\! D \!
\left( \rho \right) \,\mathrm{d}\rho , \label{eq.difflengthintegration}\end{equation}where  $\rho_o$ is the surface density. Under the assumption that the diffusivity coefficient 
$D\left(\rho\right)$ and thedensification rate   $\frac{\mathrm{d}\rho}{\mathrm{d}t}$ are known,integration from surface density $\rho_o$ to the close--off density $\rho_{co}
$can be performed yielding a model based estimate for the diffusion length.In this work we make use of the Herron--Langway densification model (\textbf{H--L 
hereafter})and the diffusivity rate parametrisation introduced by \cite{Johnsen2000} (see \textbf{SOM}).In Fig. \ref{fig:diffusion_length_profiles} \begin{figure}[t]	
\vspace*{2mm}	\begin{center}		\includegraphics[width=0.6\textwidth]{diffusion_length_profiles}		\caption{Diffusion length profiles} 		
\label{fig:diffusion_length_profiles}	\end{center}\end{figure}%%%%%%%%%%%%%%%%%%%%%%%%%%%%%%%%%%%%%%%%%%%%%%%%%%%%%%%
%%%%%%%%%%%%%%%%%%%%%%%%%%%%%%%%%%%%%%%%%%%%%\section{The single isotopologue diffusion}The simplest implementation of the 
water isotope diffusion thermometer focuses on theassessment of the diffusion rates of one isotopologue in firn.Implementation of eq. \ref{eq.difflengthintegration}, yields 
diffusion length values from surface to close off and below with ice thinning disregarded for now. A fundamental property of the convolution operation is that it is equivalent 
to  multiplication in the frequency domain. The transfer function of the diffusion process will be given by theFourier transform of the Gaussian filter that will itself be a 
Gaussian function described by \citep{Abramowitz1964, Gkinis2014}:\begin{equation}\label{ftrans_gaussian}\mathfrak{F}[ \mathcal{G} (z) ] =\hat{\mathcal{G}} = {e}
^{\frac{-k^2 \sigma^2}{2}} {} \enspace ,\end{equation}%e10where $k = 2\pi / \Delta$ and $\Delta$ is the sampling resolution of the isotopic time series. In Fig. 
\ref{fig:transfer_function} we illustrate the effect of $\hat{\mathcal{G}}$ on different wavelengths for $\sigma = 1,\, 2,\, 4\; \mathrm{and} \;8 \;	\mathrm{cm.}$ Wavelengths 
in the order of 50 cm and above remain largely unaltered while signals with wavelengths shorter than 10 cm are heavily attenuated. An estimate of the value of the 
diffusion length $\sigma$ can be obtainedby looking at the power spectrum of the diffused isotopic time series.\begin{figure}[t]	\vspace*{2mm}	\begin{center}		
\includegraphics[width=0.6\textwidth]{transfer_function}		\caption{caption transfer function} 		\label{fig:transfer_function}	\end{center}\end{figure}\begin{figure}[t]	
\vspace*{2mm}	\begin{center}		\includegraphics[width=0.6\textwidth]{synthetic_power_spectras_single}		\caption{caption synthetic power spectra} 		
\label{fig:synthetic_power_spectra}	\end{center}\end{figure}%%%%%%%%%%%%%%%%%%%%%%%%%%%%%%%%%%%%%%%%%%%%%%%%%%%%%%
%%%%%%%%%%%%%%%%%%%%%%%%%%%%%%%%%%%%%%%%%%%%%%\section{The differential diffusion signal}Section on the differential diffusion 
signal.%%%%%%%%%%%%%%%%%%%%%%%%%%%%%%%%%%%%%%%%%%%%%%%%%%%%%%%%%%%%%%%%%%%%%%%%%%%%%%%%%
%%%%%%%%%%%%%%%%%%%\section{The diffusion length ratio}Section on the ratio of diffusion lengths.%%%%%%%%%%%%%%%%%%%%%%%%%%%%
%%%%%%%%%%%%%%%%%%%%%%%%%%%%%%%%%%%%%%%%%%%%%%%%%%%%%%%%%%%%%%%%%%%%%%%%\section{The linear 
correlation method}Section on the linear correlation method.%%%%%%%%%%%%%%%%%%%%%%%%%%%%%%%%%%%%%%%%%%%%%%%%%%%%%%
%%%%%%%%%%%%%%%%%%%%%%%%%%%%%%%%%%%%%%%%%%%%%%\section{Previous theory section by Christian}%The diffusion theory used in 
this study is based on the work outlined in \citet{Johnsen1977,Johnsen2000,vanderWel2015a}. As this paper examines the methodical procedure of the different proxies, 
the authors found it necessary to outline the general theory behind the proxies.\\%After the snow has been deposited on the ice (330-360 $\mathrm{kgm}^{-3}$), it starts 
transforming into glacier ice (917 $\mathrm{kgm}^{-3}$) as the new overlying snow compresses it \citep{Herron}. During this densification process, the water stable 
isotope signal experiences a strong attenuation due to molecular diffusion \citep{Johnsen1977,Johnsen2000}. Originally, \citet{Johnsen1977} solved the diffusion equation 
for an isotopic signal ($\delta$) which experiences compression ($\dot{\varepsilon}_z$):%\begin{multline} \label{eq:convolution}%\delta(z,t) = \frac{1}{\sigma(t) \sqrt{2\pi}} 
\int\limits_{-\infty}^{\infty} \delta\left(z'\exp\left( -\int\limits_{0}^{t}\dot{\varepsilon}_z(\tau)\mathrm{d}\tau\right),0\right)\cdot \\ %\exp\left(-\frac{1}{2}\frac{(z - z')^2}
{\sigma^2(t)}\right)\mathrm{d}z'.%\end{multline}%\begin{equation} \label{eq:convolution}%\delta(z,t) = \frac{1}{\sigma(t) \sqrt{2\pi}} \int\limits_{-\infty}^{\infty} \delta
\left(z'\exp\left( -\int\limits_{0}^{t}\dot{\varepsilon}_z(\tau)\mathrm{d}\tau\right),0\right)\cdot \exp\left(-\frac{1}{2}\frac{(z - z')^2}{\sigma^2(t)}\right)\mathrm{d}z'.%
\end{equation}%Here $\sigma$ is the average vertical displacement of a water molecule defined as the diffusion length. \\%Equation \eqref{eq:convolution} is equivalent 
to the mathematical concept of convolution:%\begin{equation}%\delta(z,t) = \delta(z,0) \ast \mathcal{G},%\end{equation}which corresponds to a smoothing of the profile 
with a Gaussian filter ($\mathcal{G}$) with standard deviation $\sigma$:\begin{equation} \label{eq:Gaussian}\mathcal{G} = \frac{1}{\sigma \sqrt{2 \pi} } e^{\frac{- z^2}{2 
\sigma^2}}.\end{equation}\citet{Johnsen1977} found that the diffusion length can be computed by solving the relation between the diffusion length and the strain-rate:
\begin{equation} \label{eq:diffusion_eq}\frac{1}{2}\frac{\mathrm{d}\sigma^2}{\mathrm{d}t} - \dot{\varepsilon}_z \sigma^2 = D(t),\end{equation}where $D(\rho)$ is the 
diffusivity which applies until pore close off:\begin{equation}D(\rho) = \frac{m p D_{a}}{R T \alpha \tau}\left(\frac{1}{\rho} - \frac{1}{\rho_i}\right),\end{equation}here \textit{m} 
is the molar mass of water and $\rho_i$ is the density of glacier ice. The saturation vapor pressure ($p$) over ice is \citep{Murphy2006}:\begin{equation}p = \exp(9.5504 - 
5723.265/T + 3.530\ln(T) - 0.0073T),\end{equation}with T being the ambient temperature in Kelvin. The fractionation factors for the vapor-ice transition are here defined 
by \cite{MerlivatandNief,Majoube1970}:\begin{align} \label{eq:frac_O18}&\alpha_{O18} = 0.9722\exp(11.839/T),&\\&\alpha_D = 0.9098\exp(16289/T^2).\end{align}The 
diffusivity of water vapor in air $\left[\mathrm{m}^2/\mathrm{s}\right]$ is described as \citep{Hall}:\begin{equation}D_{ai}  = 2.1\cdot 10^{-5}\left(\frac{T}
{T_0}\right)^{1.94}\left(\frac{P_0}{P}\right).\end{equation}where $T_0 = 273.15 \mathrm{K}$ and $P_0 = 1 \mathrm{atm}$, and \textit{P} is the ambient pressure in 
atmosphere. The tortuosity factor ($\tau$) define the shape of the open channels in the firn and is assumed to only be dependent on the density \citep{Schwander1988}:
\begin{equation}\frac{1}{\tau} = 1 - b\left(\frac{\rho}{\rho_i}\right)^2, \indent b =1.3\end{equation}And for the specific water species \citep{MerlivatandJouzel}:
\begin{equation}D_a  = \begin{cases} \frac{D_{ai}}{1.0285} & 	\text{for } \delta^{18}\mathrm{O}, \\\frac{D_{ai}}{1.0251}& \text{for }     \delta \mathrm{D},  %\end{cases}
\end{equation}Based on this, the solution to Eq. \eqref{eq:diffusion_eq} can be written to be dependent on the density ($\rho$):\begin{equation} \label{eq:firn_diff}
\sigma^2(\rho) = \frac{1}{\rho^2} \int^\rho_{\rho_0} 2 (\rho')^2 D(\rho')\left(\frac{\mathrm{d}\rho'}{\mathrm{d} t}\right)^{-1} \mathrm{d}\rho'.\end{equation}Combing Eq. 
\eqref{eq:firn_diff} with a \citet{Herron} density model that is temperature and accumulation rate dependent as in \citet{Gkinis2014}, the diffusion length for a given 
temperature and accumulation rate can be calculated. \\The amount of diffusion exposed to the deposited signal can be determined by examining the power spectral 
density (PSD) of the measured signal. The maximum entropy method (MEM) developed by \cite{Burg} has proven to be a reliable way of determining the PSD of water 
stable isotope measurements \citep{Simonsen2011,Gkinis2014}. As a result, the MEM procedure has been used in this study by implementing an algorithm described in 
\citet{Andersen1974}.\\\noindent After the PSD of the data has been computed, it is possible to estimate the diffusion length by fitting the PSD of Eq. 
\eqref{eq:convolution}. This is done by taking the Fourier transform of Eq. \eqref{eq:Gaussian} which allows a transformation into the frequency domain. The convolution 
theorem then makes it possible to separate the different parts:\begin{align}\mathfrak{F}\left[\delta(z,t)\right] = &\mathfrak{F}\left[\delta(z,0)\right]\cdot \mathfrak{F}
\left[\frac{1}{\sigma \sqrt{2 \pi}}e^{-z^2/(2\sigma^2)}\right] \\ \mathfrak{F}\left[\delta(z,t)\right]=&\mathfrak{F}\left[\delta(z,0)\right]\cdot e^{-\frac{1}{2}(2 \pi f \sigma)^2}.
\end{align}This can then be expressed as the PSD by squaring the Fourier transform:\begin{equation} \label{eq:gaussfilter}P_\sigma(f) = P_0 e^{-(2\pi f \sigma)^2},
\end{equation}where $P_0$ is the PSD when then snow was deposited. The frequency range is defined by the Nyquist frequency $f\in [0, \frac{1}{2\Delta}]$ with $\Delta$ 
being the sample resolution. A noise model needs to be added to the signal in order to compensate for various noise induced effects:\begin{equation} 
\label{eq:PSDmodel}P_s(f) = P_{\sigma} + |\hat{\eta}(f)|^2,\end{equation}where $|\hat{\eta}(f)|^2$ is a red noise model described by an autoregressive (AR-1) model of 
order 1 \citep{Kaye}:\begin{equation}|\hat{\eta}(f)|^2 = \frac{\sigma^2_\eta \Delta}{|1 + a_1 \exp\left(-i 2 \pi f \Delta\right)|^2}.\end{equation}Here $\sigma^2_\eta$ is the 
variance of the noise and $a_1$ is the autoregressive coefficient defined in interval [0.1;0.3] depending on the data set. \citet{Gkinis2014} showed that an AR-1 model 
could withstand the influence from short or long term memory of the $\delta^{18}$O time series due to climate variability. They furthermore showed that the derived 
diffusion lengths with this method were accurate and insensitive to the sampling scheme. \\Based on this, the diffusion length can be estimated by adjusting $\sigma^2$, 
$P_0$, $\sigma^2_\eta$ and $a_1$ such that the fit between the model and the data gets minimized. In order to convert this spectral derived diffusion length (now $
\hat{\sigma}^2$) to a firn diffusion length, it needs to be corrected for sampling diffusion ($\sigma_{spl}$), ice diffusion ($\sigma_{ice}$) and thinning (\textit{S}(\textit{z})) of 
the layers:\begin{equation}\sigma^2_{firn} = \frac{\hat{\sigma}^2 - \sigma_{spl}^2 - \sigma_{ice}^2}{S(z)^2}.\end{equation}A sampled isotope record gives rise to an 
aliasing effect in the spectrum ($\sigma^2_{spl}$). In order to correct for this aliasing effect, the discrete measurements are represented by the transfer function of a 
rectangular filter with width $\Delta$ as in \citet{Simonsen2011}, and the continuous measurements are represented by the transfer function of a Gaussian distribution as 
in \citet{VasPhd}. The ice diffusion ($\sigma^2_{ice}$) is estimated from the borehole temperatures, but the effect should be minimal as all the ice spans the Holocene 
period due to low borehole temperatures and a short deposition time.The thinning for each core is computed by using a steady state \citet{DJmodel} flow model. Based on 
this, it is possible to estimate the integrated firn column temperature by minimizing the residual between a firn diffusion model ($\sigma'^{2}_{firn}$) and the spectral 
derived firn diffusion length ($\sigma^2_{firn}$). In other words, it finds the temperature that resulted in the observed amount of firn diffusion. The firn diffusion length is 
found by integrating Eq. \eqref{eq:firn_diff} down to pore close off ($\rho_{pc}$). The minimization is done with the use of the Newton-Raphson method that estimates the 
root of the term:\begin{equation}\left(\frac{\rho_{pc}}{\rho_i} \right)^2  \sigma^2(\rho = \rho_{pc},T(z), A(z)) - \sigma^2_{firn},\end{equation}where \textit{A}(\textit{z}) is a 
known property.\\The above procedure outlines how the single isotope diffusion lengths ($\sigma^2_{O18}$, $\sigma^2_D$) are estimated and used as temperature 
proxies. The same method can be used when examining the difference in firn diffusivity. As the $\delta^{18}$O signal experiences more diffusion than $\delta$D 
\citep{Johnsen2000}, the differential diffusion length $\Delta \sigma^2 = \sigma^2_{O18} -\sigma^2_{D}$ also has the properties of a temperature proxy. The differential 
diffusion length has the advantage of being independent of sampling and ice diffusion as the applied sampling scheme and the ice diffusion affect both isotopes equally.  \
\This study discusses three ways of computing this proxy. The first and most obvious way is to compute the single isotope diffusion lengths separately as above and 
subtract them. Secondly, the $\Delta \sigma^2$ can be computed by examining the PSD ratio of $\delta^{18}$O and $\delta$D:\begin{equation}\frac{P_D}{P_{18}} = 
\frac{P_{0_D} e^{-(k \sigma_{D})^2}}{P_{0_{O18}} e^{-(k\sigma_{O18})^2}},\end{equation}where $k=2\pi f$. This was proposed by \cite{Johnsen2000} as $P_{18}$ and 
$P_D$ increasingly deviate from each other up to the annual peak. Taking the logarithm of the expression yields:\begin{equation} \label{eq:diff_regression}\ln 
\left(\frac{P_D}{P_{18}}\right) = k^2\left( \sigma^2_{O18} - \sigma^2_{D}\right) + \ln\left(\frac{P_{0_D}}{P_{0_{O18}}}\right) =k^2 \Delta \sigma^2  + C,\end{equation}where 
$C = \ln\left(\frac{P_{0_D}}{P_{0_{O18}}}\right)$. $\Delta \sigma^2$ and $C$ can then be estimated by applying linear regression to the signal part of the $k^2$-space, 
hence excluding the noise. However, this method requires a cutoff frequency in the $k^2$-space in order to distinguish between the signal and noise domain. In this study, 
a manual examination has been performed where the different sections have been analysed separately. The cutoff frequency is thus based on the visual transition 
between the signal and the noise of the $P_{18}$.\\The third procedure used here was published in \cite{vanderWel2015a}. Here they computed the differential diffusion 
length based on the principle that the Pearson correlation coefficient between $\delta^{18}$O and $\delta$D had its maximum before diffusion took place, and it was very 
close to 1. As a consequence of $\delta^{18}$O experiencing more diffusion than $\delta$D, the correlation coefficient decreases according with the diffusion. Therefore, 
the $\Delta \sigma^2$ is estimated by artificially diffusing the $\delta$D signal with a Gaussian filter (as in Eq. \eqref{eq:Gaussian}) until it reaches a maximum correlation 
with $\delta^{18}$O. In this case, the added diffusion corresponds to $\Delta \sigma^2$.\\\noindent The last diffusion-based temperature proxy presented here is based on 
the ratio of the two single diffusion lengths. \citet{Johnsen2000} found that the ratio between the analytically estimated firn diffusion lengths only were dependent on the 
fractionation factors and hereby temperature:\begin{equation}\frac{\sigma^2_{18}}{\sigma^2_D} = \frac{1.0285 \alpha_D}{1.0251 \alpha_{18}} = 0.9327 \cdot e^{16288/T^2 
- 11.839/T}.\end{equation}The constants in front of the fractionation factors are specific for the given isotopes \citep{MerlivatandJouzel}.Since the ratio only depends on 
temperature, it has the properties to become a temperature proxy.\\This proxy has the advantage of being independent on the thinning, but dependent on the sampling 
and ice diffusion:\begin{equation} \frac{ \hat{\sigma}^2_{18} - \sigma^2_{dis} - \sigma^2_{ice} }{\hat{\sigma}^2_{D} - \sigma^2_{dis} - \sigma^2_{ice} }.\end{equation}%%%
%%%%%%%%%%%%%%%%%%%%%%%%%%%%%%%%%%%%%%%%%%%%%%%%%%%%%%%%%%%%%%%%%%%%%%%%%%%%%%%%%%%%
%%%%%%%%%%%%%\section{Synthetic data}It is essential to verify the performance of the diffusion-based computations in order to be ascertain that the estimated 
signal really represents the imposed firn diffusion. This has been accomplished by generating series of synthetic data, which will demonstrate the accuracy and precision 
of the different proxies.The synthetic data have to represent different climatic conditions, in order to completely assess the viability of the proxies. Therefore, two types of 
synthetic data sets have been constructed with one simulation representing a "warm" and humid drill (WH) site such as NEEM ($T = -29.0\,\mathrm{C}$, $A = 0.22\, 
\mathrm{myr}^{-1}$) and one simulating a cold and dry drill (CD) site such as Dome C ($T = -55.0\,\mathrm{C}$, $A = 0.032\, \mathrm{m/yr}$). \\The high resolution $
\delta^{18}$O time series are generated by assuming an AR-1 process identical to that of \cite{Gkinis2014}. The process is applied on Gaussian noise with a mean of 
zero and variances of $120\,\permil^2$ and $200\,\permil^2$ for respectively the CD and the WH drill sites. Each time series have a spacing of $\Delta x = 10^{-3} 
\mathrm{m}$ and a total length of $20\,\mathrm{m}$. A corresponding $\delta$D series is then computed from the global meteoric water line:\begin{equation} \delta
\mathrm{D} = 8 \cdot \delta^{18}\mathrm{O} + 10 \permil.\end{equation}The two isotopic series are then convolved with Gaussian filters of predefined variances, 
simulating the effect of firn diffusion for the specific isotope. A sampling scheme with a resolution of $\Delta = 0.025\,\mathrm{m}$ representing the width of high resolution 
discrete measurements is then applied on the diffused series. Finally, white noise is added to the sampled records with a standard deviation of $0.07\,\permil$ for $
\delta^{18}$O and $0.50\,\permil$ for $\delta$D in order to simulate measurement noise.\\Figures \ref{fig:WH_spectra} and \ref{fig:CD_spectra} show the spectra for the 
WH and CD scenarios. Each scenario has been simulated 100 times, and the computed average diffusion length and root-mean-square (RMS) values have been and 
displayed in Table \ref{tbl:synthetic_diff_lens}. Furthermore, as the purpose of using firn diffusion as a paleoclimatolic thermometer is to reconstruct accurate and precise 
temperatures, the corresponding temperatures and uncertainties that match the derived diffusion lengths are shown.\\The synthetic data demonstrate that the all of the 
presented methods are successfully able to derive the original diffusion lengths which convolved the initial signal. Furthermore, it is evident that the methods provide 
accurate and precise results independent of the drill site simulations. The highest uncertainty is found by the  $\Delta\sigma^2$ estimated from subtraction and the $
\sigma^2_{18}/\sigma^2_D$ for the CD scenario. Nonetheless, both proxies still perform with a high accuracies.\begin{figure}[t]	\vspace*{2mm}	\begin{center}		
\includegraphics[width=0.5\textwidth]{NEEM_spectra}		\caption{The WH data. Upper plot: Blue represents the logarithm of the PSD ratio with respect to the $k^2$-
space. The black curve shows the fit in the signal part of the domain with the slope being the $\Delta\sigma^2$. Lower plot: The PSD of $64\cdot \delta^{18}$O (blue) and 
the $\delta$D (red) data.} \label{fig:WH_spectra}	\end{center}\end{figure}\begin{figure}[t]	\vspace*{2mm}	\begin{center}		\includegraphics[width=0.5\textwidth]
{DOMEC_spectra}		\caption{The CD data. Upper plot: Blue represents the logarithm of the PSD ratio with respect to the $k^2$-space. The black curve shows the fit in 
the signal part of the domain with the slope being the $\Delta\sigma^2$. Lower plot: The PSD of $64\cdot \delta^{18}$O (blue) and the $\delta$D (red) data.} 
\label{fig:CD_spectra}	\end{center}\end{figure}\begin{table*}[t]	\caption{Table showing the mean and RMS error of the computed diffusion lengths together with the 
corresponding temperatures for the CD and WH simulations. The WH is simulated with an annual temperature of $-29.0\,\mathrm{C}$ and an accumulation rate of $0.22\, 
\mathrm{myr}^{-1}$ and the CD is simulated with an annual temperature of $-55.0\,\mathrm{C}$ and an accumulation rate of $0.032 \,\mathrm{myr}^{-1}$. $\Delta\sigma 
\text{ I }$ denotes subtraction, $\Delta\sigma \text{ II }$ denotes the logaritmic fit, and $\Delta\sigma \text{ III }$ denotes the correlation-based method.}
\label{tbl:synthetic_diff_lens}	\begin{tabular}{l l l l  l l l } 		%\tophline		& &  CD & &  &WH&\\	%	\middlehline		Proxy  &	Real diff. length	&	Est. diff. length	& Est. 
Temp. &	Real diff. length	&	Est. diff. length	& Est. Temp.\\		%\middlehline			$\sigma_{18}$	&	 $6.76$	&	$6.87 \pm 0.24$	 &  $-54.9  \pm 0.7 $ 	&	$9.45$	&	
$9.53 \pm 0.31$	& $-28.8 \pm 0.8 $\\ 		$\sigma_{D}$&	$5.93$	&$5.99 \pm 0.14$	 &	$-54.9  \pm 0.5$ &	$ 8.57$	&	$8.63 \pm 0.29$	& 	$-28.8 \pm 0.9$\\ 		$\Delta
\sigma^2$ I &$ 10.53	$&	$11.3 \pm 2.7$	& $-54.8  \pm 2.9$   &	$15.86$	&	$16.5 \pm 4.3$ 	&$-28.7 \pm 4.3$\\		$\Delta\sigma^2$ II& $10.53$	&	$10.7 \pm 
1.0$		& $-55.4  \pm 1.1$	&	$ 15.86$	&$16.1 \pm 1.9$	& $-29.1 \pm 1.9$\\		$\Delta\sigma^2$ III& $10.53$	&	$10.4 \pm 0.6$		&  $-55.8  \pm 0.7$	&	$15.86$
	&	 $16.3 \pm 1.0$& $-28.9 \pm 1.0$\\		$\sigma^2_{18}/\sigma^2_D$&$	1.30$& $1.32 \pm 0.08$ 	& $-56.0  \pm 7.3$	&$1.22$	& 	$1.18 \pm 0. 06$ 	& $-28.7 
\pm 0.9$\\		%\bottomhline			\end{tabular}\end{table*}\section{The ice core data}This study has compiled a series of ice core sections from Greenland and Antarctica 
which are displayed in Table \ref{tbl:drill_sites}. The present day annual temperatures vary between $-29\,\mathrm{C}$ and $-55\,\mathrm{C}$ together with accumulation 
rates between $3\, \mathrm{cmy}^{-1}$ to $23 \,\mathrm{cmy}^{-1}$.The selected ice core sections are divided into two groups such that the first group represent the late 
Holocene period and the second group represent the early Holocene period.The late Holocene ice cores were selected from the criteria that the age should be close to 
present day with respect to time, and such that both $\delta^{18}$O and $\delta$D data were available. This makes it possible to better compare the different results with 
a reference temperature.\\The early Holocene group serves to further investigate the sensitivity and performance of the different proxies. A NEEM ice core section is here 
presented, that have been measured both discretely ($5 \,\mathrm{cm}$) and continuously ($0.50\, \mathrm{cm}$), which makes it possible to assess if the sampling has 
an impact on the proxies. Two NGRIP (I and II) ice cores were drilled as the first drill got stuck, and here the overlapping section of the two cores are used. Even though 
the overlapping section might not completely represent the same age (\textbf{probably 8 meters, source?}), the overlap can help determine the influence from natural 
variability with this method. An early Holocene section in the GRIP ice core is used in this study in order to show any irregularities compared to the late Holocene GRIP 
data. Two Antarctic ice core sections are here used, where Dome F was measured continuously and Dome C discretely and both drill sites have low accumulation 
rates.All the presented ice cores here except for NGRIP have dual isotope measurements. \\All the ice cores span the Holocene period, and the spatial variability is well 
represented. Drill sites with annual temperatures warmer than $-29 \mathrm{C}$ were not chosen for this study as these drill sites experience a vast amount of summer 
melting. This has a significant impact on the diffusion because the pore close off depth can be in the top layer of the ice due to melt layers, hereby making this procedure 
unreliable.\begin{table*}[t]	\caption{Table showing the different ice cores together with their average age [b2k] and the corresponding drill site characteristics. The GRIP, 
NEEM and NGRIP core were dated with respect to the GICC05 time scale. The EMDL data was published in \cite{Oerter2004}, and the reconstructions for accumulation 
rate and total thinning for EDML are provided by the Antarctic ice core Chronology \citep{Veres2013}.}\label{tbl:drill_sites}	\begin{tabular}{l l l l l l l l l} 		%\tophline		Ice 
cores &Depth [m]&Res. [cm]& Age [kyr] & Annual T [C] & A [m ice/yr] & P [atm] & Thinning& Meas. \\		%\middlehline				GRIP mid&$753-776$ &2.5& 3.7 &$-31.7 $ 
& 0.23 & 0.65 & 0.72&$\delta$D, $\delta^{18}$O\\		GRIP late&$514-531$ &2.5 &2.4 &$-31.7 $ &0.23 & 0.65  &  0.79&$\delta$D, $\delta^{18}$O\\		 		EDML & 
$123-178$&5.0&1.6&$-44.6$ & 0.08 & 0.67 & 0.93&$\delta$D, $\delta^{18}$O\\				NEEM&$174-194 $&2.5&0.8 &$-29.0$ & 0.22 &  0.72 &  0.31&$\delta$D, $
\delta^{18}$O\\       		NGRIP &$174-194$ & 2.5&0.9 &$-31.5$ & 0.20 & 0.67  & 0.49& $\delta^{18}$O\\					%\middlehline		Dome F& $302-307$&0.5& 9.6 & 
$-55.0$ &0.04 & 0.61  &  0.93 &$\delta$D, $\delta^{18}$O\\		Dome C &$308-318$ &2.5& 9.9 &$-55.0 $ &0.04 & 0.65 &   0.93&$\delta$D, $\delta^{18}$O\\		GRIP 
early&$1449-1466$ &2.5& 9.4 &$-31.6 $ & 0.20 & 0.65  &  0.45&$\delta$D, $\delta^{18}$O\\		NEEM dis &$1380-1392 $&5.0&10.9 &$-29.0$ & 0.22 &  0.72 &  0.31&$
\delta$D, $\delta^{18}$O\\       		NEEM CFA& $1382-1399$&0.5& 10.9 &$-29.0 $& 0.22 & 0.72  & 0.31&$\delta$D, $\delta^{18}$O\\					NGRIP 1&$1300-1320$ 
& 2.5&9.1 &$-31.5$ & 0.20 & 0.67  & 0.49& $\delta^{18}$O\\					NGRIP 2 &$1300-1320$ &2.5& 9.1 &$-31.5$ & 0.20 & 0.67  & 0.49&$\delta^{18}$O\\					
	%\bottomhline			\end{tabular}\end{table*}\section{The annual peak}Drill sites with relatively high accumulation rates (>20 cm ice/yr) and a low vertical thinning of 
the layers, will have a prominent annual peak in the power spectrum. For cores such as NEEM and NGRIP this is evident in the top 300 m of the ice, whereas its evident 
below 770 m for GRIP. Figure \ref{fig:filter_function} shows the PSD of a $\delta$D series (GRIP, mid Holocene). Here the annual peak is visible around 6 cycles/m, and it 
is evident that it has a high PSD compared to the surrounding frequencies. A strong annual peak will artificially decrease the spectral derived diffusion length when fitting 
the model (Eq. \eqref{eq:PSDmodel}), hereby resulting in colder temperatures. As a result, this study proposes the incorporation of a frequency weighted filter in the 
spectrum, thus completely removing the influence from the annual peak. If the annual peak ($\lambda_{freq}$) is persistent in the spectrum, it can be found by manually 
examining the individual spectra. It can also be found by an automatic procedure, which is relevant when applying the method on longer ice core sections:
\begin{equation}\lambda_{freq} = \frac{1 \text{ cycle}}{A [\text{m/yr}] \cdot 1 \mathrm{yr} \cdot S},\end{equation}or if the an annual layer chronology is available:
\begin{equation}\lambda_{freq} = \frac{1 \text{ cycle}}{\lambda [\text{m}]}.\end{equation}The filter is then defined as:\begin{center}	filter $=  \begin{cases} 0 &  f \in 
[\lambda_{freq} - 0.5;\lambda_{freq} + 3.0],	 \\								1                           & \text{else }          .   %	\end{cases}$\end{center}This weight function is multiplied 
with the cost function between the data based PSD and Eq. \eqref{eq:PSDmodel}.\\How far the filter should span into the higher frequency domain is difficult to assess, as 
it is essential not to introduce a bias by excluding information about past climate. In general, this method is more reliable for multiannual, decadal variability and long term 
climate change, as the firn diffusion smoothens out the high frequencies first. Therefore, it is important to include all of the signal before the annual peak. For frequencies 
higher than this, it is difficult to quantify whether the remaining of this red noise transition is a climatic signal, measurement diffusion, noise induced and/or a spectral 
artifact of the MEM. It is relevant to note that if it is climate signal with a higher frequency than the annual peak, then it is subannual signal. This method should be 
preferred over a cutoff frequency, as this procedure exclude less information.Moreover, an examination of Fig. \ref{fig:filter_function} reveals that by excluding the annual 
peak, the fit provides a better match to the data. For comparison, the spectral derived diffusion lengths shown in the Fig. \ref{fig:filter_function} are 4.78 cm and 5.08 cm 
for respectively the unmodified and filtered spectra.\\The implementation of this filter will not only improve the single isotope diffusion length proxies but also the $\Delta 
\sigma^2$ proxy derived from subtraction and the $\sigma^2_{18}/\sigma^2_D$ proxy as the annual peak is more persistent in the $\delta$D record. The annual peak will 
therefore affect the two isotopes differently. \\This procedure is unnecessary in most cases, as the annual peak smoothens out deeper down in the ice. GRIP is therefore a 
special case due to the combination of a high accumulation rate and not too rapid a thinning of the layers.  \begin{figure}[t]	\vspace*{2mm}	\begin{center}		
\includegraphics[width=0.5\textwidth]{weighted_filter_GRIP}		\caption{GRIP mid Holocene spectrum for $\delta$D showing the used weights. The black rectangular box 
shows the weighted filter, the red curve represents the computed PSD and the blue curves represent the estimated signals and the brown curves represent the noise. 
Here the solid lines represent including the annual peak and the dashed lines represent excluding the annual peak). Autoregressive order is 30.} \label{fig:filter_function}	
\end{center}\end{figure}\section{Reconstructed temperatures}The reconstructed temperatures and their corresponding uncertainties were estimated by randomly 
extracting windows out of a data section, where each window contained a different amount of samples (>400 samples). This method quantifies how large an impact the 
neighboring values have on the derived temperature. So the uncertainties presented here does not reflect uncertainties in the densification model. This procedure was 
performed 1000 times for each data set, and the errorbars are defined as 1$\sigma$. \subsection{The late Holocene}Figure \ref{fig:all_temps} shows the derived 
temperatures using the different diffusion length proxies plotted together with the annual drill site temperature (borehole temperature). The derived temperatures are not 
expected to match consistently with the measured temperatures, as the climate has a natural variability. Thus in order to make an accurate comparison, averaging over a 
long record is essential. Furthermore, the aim of this study is to compare the proxies with each other and examine the variability and precision of the outputs. 
Nonetheless, as all the cores represent the stable Holocene period, then only minor deviations are expected. \\The performance of the single $\sigma^2_{18}$ and $
\sigma^2_{D}$ proxies can be examined from Fig. \ref{fig:all_temps}. The two proxies reconstruct similar temperatures for all the ice cores, and they have a high precision 
and consistency. It is not possible to completely compare the accuracies of the $\sigma^2_{18}$ and $\sigma^2_{D}$ proxies with the other proxies. This is a 
consequence of the proxies likely representing relative temperature signals. Thus, the proxies averaged over a long record should result in similar temperatures, 
especially with respect to climate transitions such as the Holocene-glacial transition.\\Nevertheless, it is intriguing that despite slight discrepancies between the single and 
differential diffusion lengths, the two $\sigma^2$ proxies are consistent with each other, and the $\Delta\sigma^2$ proxies from subtraction and log-fit reconstruct coherent 
temperatures.\\In general, the $\Delta\sigma^2$ proxies have higher uncertainties compared to the single isotope proxies, even though theory predicts them to be more 
reliable due to sampling and ice diffusion canceling out.\\The EMDL temperatures have very low uncertainties which likely are a result of the low accumulation rate and 
temperature making this site ideal for these proxies. \citet{vanderWel2015a} carried out a diffusion study on the EDML core and performed temperature reconstructions 
using different varieties of diffusion proxies including the $\sigma^2_{18}$ and $\sigma^2_{D}$. They found similar temperatures for the section of the ice core used in this 
study. Furthermore, they expanded their analysis and applied the method to a longer section, where they found that the average temperatures for the $\sigma_{18}^2$, $
\sigma_D^2$ and $\Delta\sigma^2$ proxies were in agreement with the annual drill site temperature despite the individual results for this section being contradicting.
\cite{vanderWel2015a} also developed the new correlation-based $\Delta\sigma^2$ proxy. As Fig. \ref{fig:all_temps} shows, this proxy fails to accurately and precisely 
determine past temperatures for the different drill sites except for the EDML core. The same applies for the ${\sigma^2_{18}}/{\sigma^2_D}$ proxy which simply has to 
large uncertainties to conclude anything useful. The ${\sigma^2_{18}}/{\sigma^2_D}$ ratio might fail as a proxy due to the poorly understood fractionation factors, which 
will be tested and discussed in Sect. \ref{sec:fractionation}.\\\begin{figure*}[t]	\vspace*{2mm}	\begin{center}		\includegraphics[width=\textwidth]{late_holo_temps_new}	
	\caption{Temperature reconstructions based on the diffusion lengths of different ice cores where the black stars represent the annual mean temperature at the drill 
sites. The left figure shows all the proxies together, where the right figure shows a zoomed in section of the proxies. Blue circles represent $\sigma^2_{18}$, red circles 
represent $\sigma^2_{D}$, the $\Delta \sigma^2$ proxy found by subtraction is represented by the green squares, the logarithmic fit is represented by the brown colored 
squares, the correlation-based method is represented by the magenta squares and the ${\sigma^2_{18}}/{\sigma^2_D}$ proxy is represented by the grey colored 
triangles.}  \label{fig:all_temps}	\end{center}\end{figure*}\subsection{The early Holocene}%\begin{table*}[t]%	\caption{Table showing the different ice cores together with 
their approximated age [B.P] and the corresponding drill site characteristics. The GRIP, NEEM and NGRIP core were dated with respect to the GICC05 time scale. 
\textbf{Cite the source of the other cores}}\label{tbl:other_drill_sites}%	\begin{tabular}{l l l l l l l l l} %		\tophline%		Ice cores &Depth [m]&Res. [cm]& Age [kyr] & Annual 
T [\unit{\degree C}] & A [m ice/yr] & P [atm] & Thinning& Meas. \\%		\middlehline%		Dome F& $302-307$&0.5& 9.6 & $-55.0$ &0.04 & 0.61  &  0.93 &$\delta$D, $
\delta^{18}$O\\%		Dome C &$308-318$ &2.5& 9.9 &$-55.0 $ &0.04 & 0.65 &   0.93&$\delta$D, $\delta^{18}$O\\%		GRIP early&$1449-1466$ &2.5& 9.4 &$-31.6 $ & 
0.20 & 0.65  &  0.45&$\delta$D, $\delta^{18}$O\\%		NEEM dis &$1380-1392 $&5.0&10.9 &$-29.0$ & 0.22 &  0.72 &  0.31&$\delta$D, $\delta^{18}$O\\       %		
NEEM CFA& $1382-1399$&0.5& 10.9 &$-29.0 $& 0.22 & 0.72  & 0.31&$\delta$D, $\delta^{18}$O\\			%		NGRIP 1&$1300-1320$ & 2.5&9.1 &$-31.5$ & 0.20 & 0.67  
& 0.49& $\delta^{18}$O\\			%		NGRIP 2 &$1300-1320$ &2.5& 9.1 &$-31.5$ & 0.20 & 0.67  & 0.49&$\delta^{18}$O\\		%		\bottomhline		%	\end{tabular}%
\end{table*}Figure \ref{fig:early_temps} shows reconstructed temperatures for ice core sections representing the early Holocene. In these cases, it is difficult to compare 
the reconstructed temperatures with borehole temperatures as no precise reference temperatures are available. However, the age of the ice cores correspond to the 
Holocene Climatic Optimum (HCO). \citet{Dorthe1998} found, based on the borehole thermometry of the GRIP ice core, that the surface temperatures were 1-3 C warmer 
during the HCO (5-8 kyr). \cite{Vinther2009} found the same HCO for multiple Greenlandic ice cores which spanned the period 5-9 kyr. The same HCO were found in 
Antarctic ice cores, where temperatures 1 C warmer than present were found 6-10 kyr ago \citep{Cias1992}.\\% It is therefore plausible that the HCO are the cause of the 
$\sigma^2_{18}$ and $\sigma^2_{D}$ proxies resulting in warmer than present day temperatures.\\For NGRIP, \cite{Gkinis2014} found that the derived early Holocene 
temperatures ($\sigma^2_{18}$) were 10 C warmer than present day, which they found to be an artifact of the accumulation rate model. As a result, an accumulation rate 
reduction of 10 \% where needed in order to obtain a 3 C HCO. This reduction was introduced both in the computation of the thinning function and the firn diffusion length. 
For GRIP, the same deviation was found for $\sigma^2_{18}$ and $\sigma^2_{D}$, thus an accumulation rate reduction of 14 \% was needed in order to retain a HCO of 3 
C . Hence, the GRIP temperatures were reconstructed by implementing a thinning function correction to the \citet{DJmodel} model identical to that of \citet{Gkinis2014}.\\
%The two overlapping NEEM sections seems to have warmer temperatures for the $\sigma^2_{18}$ and $\sigma^2_{D}$ proxies as well, hence, the need of a lower 
accumulation rate might be relevant too, but a more thoroughly analysis needs to be performed first.\begin{figure*}[t]	\vspace*{2mm}	\begin{center}		
\includegraphics[width=\textwidth]{early_holo_temps_new}		\caption{Temperature reconstructions based on the diffusion lengths of different ice cores where the black 
stars represent the annual mean temperature at the drill sites. Blue circles represent $\sigma^2_{18}$, red circles represent $\sigma^2_{D}$, the $\Delta \sigma^2$ proxy 
found by subtraction is represented by the green squares, the logarithmic fit is represented by the brown colored squares and the correlation-based method is 
represented by the magenta colored squares and the ${\sigma^2_{18}}/{\sigma^2_D}$ proxy is represented by the grey colored triangles.}  \label{fig:early_temps}	
\end{center}\end{figure*}Figure \ref{fig:early_temps} shows that the correlation-based $\Delta\sigma^2$ fails to reconstruct past climate, as it reconstructs contradicting 
results for the NEEM ice core. Here it appears to be dependent on the sampling. Furthermore, the reconstructed temperature for Dome C is well below what the other 
proxies reconstructed. Dome C has  a slightly lower accumulation rate than EDML, so the high accumulation rate of the Greenlandic cores can not be the cause of the 
previously shown discrepancy. It seems that the proxy always overestimates cold temperatures. \\\section{The fractionation factors} \label{sec:fractionation}The most 
important temperature dependence of the water stable isotopes arise from the fractionation factors. However, the fractionation factors are not completely well understood 
for low temperatures (<40 C) \citep{Ellehoj2013}, and it might have a significant influence on the derived temperatures when using the $\Delta\sigma^2$ or $
{\sigma^2_{18}}/{\sigma^2_D}$ proxies. In order to examine how great an impact the fractionation factors have on the reconstructed temperatures, temperature 
reconstructions have been performed using different fractionation factor studies.\\The previously used fractionation factors used in this study were based on 
\cite{Majoube1970,MerlivatandNief}. \cite{Ellehoj2013} presented new values of the fractionation factors, and especially the $\alpha_D$ deviated from that of 
\cite{MerlivatandNief}. The last study tested here is from \cite{Lamb2015} which measured the fractionation of $\delta$D. In this case, the \cite{Majoube1970} 
parameterization of $\alpha_{18}$ will be used for the dual diffusion length proxies.\begin{figure*}[t]	\vspace*{2mm}	\begin{center}		\includegraphics[width=\textwidth]
{fractionation_temps}		\caption{Temperature reconstructions based on the diffusion lengths of different ice cores where the black stars represent the annual mean 
temperature at the drill site. The circles represent the original used fractionation factors from \citet{Majoube1970,MerlivatandNief}, the squares are based on 
\citet{Ellehoj2013} and the triangles represent \citet{Lamb2015} $\alpha_D$ together with \citet{Majoube1970}. Blue represents $\sigma^2_{18}$, red represents $
\sigma^2_{D}$, the $\Delta \sigma^2$ proxy found by subtraction is represented by green, the logarithmic fit is represented by brown, the correlation-based method is 
represented by magenta and the ${\sigma^2_{18}}/{\sigma^2_D}$ proxy is represented by grey.}  \label{fig:fractionation_temps}	\end{center}\end{figure*}Figure 
\ref{fig:fractionation_temps} shows the derived temperatures using different fractionation factor parameterizations. It is evident that the choice of fractionation factors do 
not have any significant influence on the $\sigma^2_{18}$ and $\sigma^2_{D}$ proxies. The influence is slightly larger for the $\Delta \sigma^2$ proxies, but it is nothing 
distinguished. Whereas, the choice of fractionation factors seem to have a significant impact on the ${\sigma^2_{18}}/{\sigma^2_D}$ proxy. It is evident that the 
\citet{Ellehoj2013} parametrization decreases the uncertainty of the proxy. However, the precision and accuracy of this proxy are still not satisfactory, and this is probably 
contributed from uncertainties in the spectral derived diffusion lengths having too large an impact. The results using the \cite{Lamb2015} parameterization of $\alpha_D$ 
does not deviate significantly from that of \cite{MerlivatandNief}.\\It is noteworthy that the largest influence of the ${\sigma^2_{18}}/{\sigma^2_D}$ proxy does not arise for 
the coldest temperature at Dome C, but instead at GRIP and EDML. This supports the conclusion that this proxy is highly sensitive to uncertainties of the spectral derived 
diffusion lengths.\section{Discussion}The results presented in Fig. \ref{fig:all_temps} and \ref{fig:early_temps} showed that the single isotope diffusion lengths, $
\sigma^2_{18}$ and $\sigma^2_{D}$, resulted in the most accurate and precise reconstructed temperatures. This was evident as the two proxies reconstructed coherent 
temperatures with low uncertainties for the various drill sites. The impact of the PSD of the annual peak on the spectral derived diffusion lengths led to the proposal of 
implementing a frequency weighted filter when fitting the parameters. This filter function minimized the artificial decrease of the spectral derived diffusion lengths, but it 
was only relevant to use this filter when the annual peak was strong in the PSD of the signal.\\The differential diffusion lengths were estimated with three different 
methods. The subtraction of the single diffusion lengths and the logarithmic fit of the PSD ratio (Eq. \eqref{eq:diff_regression}) proved to be the most accurate of the three 
proxies, and it was difficult to distinguish between the most reliable of these two. The correlation-based method from \citet{vanderWel2015a} resulted in severely colder 
climate than anticipated from the annual measured temperatures. Especially prominent were the results for Dome C, GRIP (Mid) and NEEM (discrete). As of GRIP, the 
discrepancy was interesting because the same significant deviation was not found for the late and early Holocene. Moreover, the discrete measurements for NEEM 
showed a deviation of -29 C, whereas it matched well with the CFA measurements. That large a difference between cores having the same climate history and isotopic 
values proved the method to be unreliable.\\The ${\sigma^2_{18}}/{\sigma^2_D}$ proxy had an unreliable performance compared to the other proxies. As this method was 
independent of layer thinning, and with the ice and sampling diffusion being considerably small for the different ice cores, the discrepancies must be attributed to 
uncertainties in the spectral derived diffusion lengths, the diffusivities of the two isotopes and/or the fractionation factors. \\The fractionation factors from \citet{Ellehoj2013} 
and \cite{Lamb2015} were compared with \citet{Majoube1970,MerlivatandNief} in order to test the impact on the derived temperatures. It was found that the 
\citet{Ellehoj2013} parametrization had a reduced uncertainty and higher accuracy for the ${\sigma^2_{18}}/{\sigma^2_D}$ proxy, but this reduced uncertainty could still 
not make the proxy reliable.\section{Conclusions}  %% \conclusions[modified heading if necessary]The single $\sigma^2$ proxies and the $\Delta\sigma^2$ proxies 
reconstruct slightly different results in some cases. Nonetheless, it is noteworthy that all the results individually are reliable estimates for the drill sites. This is interpreted 
to be a result of the proxies having to be compared with respect to themselves over a long record, as it is the relative temperature change that is of interest (e.g. climate 
transitions). Particularly the $\Delta\sigma^2$ proxy has the potential of reconstructing accurate amplitudes of climate transitions as this proxy is independent of ice 
diffusion, which has a significant impact during the glacial. \\The $\sigma^2_{18}$ and the $\sigma^2_{D}$ proxies are superior to the dual diffusion length proxies, which 
is evident based on the low uncertainties and high accuracy of the reconstructed temperatures. The $\sigma^2_{18}$ and the $\sigma^2_{D}$ proxies are moreover less 
influenced by uncertainties in the fractionation factor parametrization, hereby making them more suitable for paleoclimatology research.%%%%%%%%%%%%%%%%%
%%%%%%%%%%%%%%%%%%%%%%%%%%%%%%%%%%%%%%%%%%%%%%%%%%%%%%%%%%%%%%%%%%%%%%%%%%%%%%%%%%%
\section*{authorcontribution}Author contributions\section*{acknowledgements}TEXT%% REFERENCES%% The reference list is compiled as follows:%
\begin{thebibliography}{}%\bibitem[AUTHOR(YEAR)]{LABEL}%REFERENCE 1%%\bibitem[AUTHOR(YEAR)]{LABEL}%REFERENCE 2%\end{thebibliography}%% Since 
the Copernicus LaTeX package includes the BibTeX style file copernicus.bst,%% authors experienced with BibTeX only have to include the following two lines:%%
\bibliographystyle{agsm}\bibliography{reference}%%%% URLs and DOIs can be entered in your BibTeX file as:%%%% URL = {http://www.xyz.org/~jones/idx_g.htm}%% 
DOI = {10.5194/xyz}%% LITERATURE CITATIONS%%%% command                        & example result%% \citet{jones90}|               & Jones et al. (1990)%% 
\citep{jones90}|               & (Jones et al., 1990)%% \citep{jones90,jones93}|       & (Jones et al., 1990, 1993)%% \citep[p.~32]{jones90}|        & (Jones et al., 1990, p.~32)%
% \citep[e.g.,][]{jones90}|      & (e.g., Jones et al., 1990)%% \citep[e.g.,][p.~32]{jones90}| & (e.g., Jones et al., 1990, p.~32)%% \citeauthor{jones90}|          & Jones et al.%
% \citeyear{jones90}|            & 1990%% FIGURES%% ONE-COLUMN FIGURES%%f%\begin{figure}[t]%\includegraphics[width=8.3cm]{FILE NAME}%\caption{TEXT}%
\end{figure}%%%% TWO-COLUMN FIGURES%%%f%\begin{figure*}[t]%\includegraphics[width=12cm]{FILE NAME}%\caption{TEXT}%\end{figure*}%%%%% TABLES%
%%%%% The different columns must be seperated with a & command and should%%% end with \\ to identify the column brake.%%%% ONE-COLUMN TABLE%%%t%
\begin{table}[t]%\caption{TEXT}%\begin{tabular}{column = lcr}%\tophline%%\middlehline%%\bottomhline%\end{tabular}%\belowtable{} % Table Footnotes%\end{table}%
%%% TWO-COLUMN TABLE%%%t%\begin{table*}[t]%\caption{TEXT}%\begin{tabular}{column = lcr}%\tophline%%\middlehline%%\bottomhline%\end{tabular}%
\belowtable{} % Table Footnotes%\end{table*}%%%%% NUMBERING OF FIGURES AND TABLES%%%%%% If figures and tables must be numbered 1a, 1b, etc. the 
following command%%% should be inserted before the begin{} command.%%\addtocounter{figure}{-1}\renewcommand{\thefigure}{\arabic{figure}a}%%%%% 
MATHEMATICAL EXPRESSIONS%%%% All papers typeset by Copernicus Publications follow the math typesetting regulations%%% given by the IUPAC Green Book 
(IUPAC: Quantities, Units and Symbols in Physical Chemistry,%%% 2nd Edn., Blackwell Science, available at: http://old.iupac.org/publications/books/gbook/
green_book_2ed.pdf, 1993).%%%%%% Physical quantities/variables are typeset in italic font (t for time, T for Temperature)%%% Indices which are not defined are 
typeset in italic font (x, y, z, a, b, c)%%% Items/objects which are defined are typeset in roman font (Car A, Car B)%%% Descriptions/specifications which are defined by 
itself are typeset in roman font (abs, rel, ref, tot, net, ice)%%% Abbreviations from 2 letters are typeset in roman font (RH, LAI)%%% Vectors are identified in bold italic 
font using \vec{x}%%% Matrices are identified in bold roman font%%% Multiplication signs are typeset using the LaTeX commands \times (for vector products, grids, and 
exponential notations) or \cdot%%% The character * should not be applied as mutliplication sign%%%%% EQUATIONS%%%% Single-row equation%%\begin{equation}
%%\end{equation}%%%% Multiline equation%%\begin{align}%& 3 + 5 = 8\\%& 3 + 5 = 8\\%& 3 + 5 = 8%\end{align}%%%%% MATRICES%%\begin{matrix}%x & y & z\\
%x & y & z\\%x & y & z\\%\end{matrix}%%%%% ALGORITHM%%\begin{algorithm}%\caption{�
}%\label{a1}%\begin{algorithmic}%�
%\end{algorithmic}%\end{algorithm}%%%%% CHEMICAL FORMULAS AND REACTIONS%%%% For formulas embedded in the text, please use \chem{}%%%% The 
reaction environment creates labels including the letter R, i.e. (R1), (R2), etc.%%\begin{reaction}%%% \rightarrow should be used for normal (one-way) chemical 
reactions%%% \rightleftharpoons should be used for equilibria%%% \leftrightarrow should be used for resonance structures%\end{reaction}%%%%% PHYSICAL UNITS
%%%%%% Please use \unit{} and apply the exponential notation\end{document}